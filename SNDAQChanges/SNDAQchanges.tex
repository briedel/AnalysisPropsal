\section{SNDAQ Changes}

The likelihood used in the SNDAQ online analysis relies on assumption that the noise rate on a per DOM level is Gaussian distributed, i.e. that the noise is purely Poissonic in nature. The per-DOM noise rate over long time scales however is known to be log-normal distributed

\begin{equation}
    f_{X}(x; \mu, \sigma) = \frac{1}{x \sigma \sqrt{2\pi}} \textrm{exp} \left\{ -\frac{1}{2} \left( \frac{ln(x) - \mu}{\sigma}  \right)^{2} \right\}
\end{equation}

\noindent where $\mu$ is the location parameter and $\sigma$ is the scale parameter. Over a short time scales as used in the SNDAQ analysis, i.e. \unit[11]{minutes}, the log-normal behavior can be approximated with a Gaussian 

\begin{equation}
    f_{X}(x; \mu, \sigma) = \frac{1}{\sigma \sqrt{2\pi}} \textrm{exp} \left\{ -\frac{1}{2} \left( \frac{x - \mu}{\sigma}  \right)^{2} \right\}
\end{equation}

\noindent where $\mu$ is the mean and $\sigma$ is the standard deviation or variance. 

Given Gaussian behavior of the DOM noise, we can measure a collective increase rate of the detector, $\Delta \mu$, employing the likelihood

\begin{equation}
    \mathcal{L}(\Delta \mu) = \prod_{i=1}^{N_{\textrm{DOM}}} \frac{1}{\sigma_{i} \sqrt{2\pi}} \textrm{exp} \left\{ -\frac{1}{2} \left( \frac{r_{i} - (\mu_{i } + \epsilon_{i}\Delta \mu)}{\sigma_{i}}  \right)^{2} \right\}
\end{equation}

\noindent where $r_{i}$ is the rate in the signal bin, $\mu_{i}$ and $\sigma_{i}$ is the average rate and error on the average rate estimated from the background region. From maximizing the log-likelihood we can obtain an analytical form of $\Delta \mu$:

\begin{equation}
  \Delta \mu = \sigma_{\Delta \mu} \sum_{i = 1}^{N_{\textrm{DOM}}} \frac{\epsilon_{i}}{\sigma_{i}} (r_{i} - \mu_{i})
\end{equation}

\noindent where 

\begin{equation}
  \sigma_{\Delta \mu} = \left( \sum_{i = 1}^{N_{\textrm{DOM}}} \frac{\epsilon_{i}^{2}}{\sigma_{i}^{2}} \right)^{-1}
\end{equation}

The assumption of purely Poissonic noise however appears to break down to do the non-Poissnic nature of the muon background on a per-DOM basis. A possible solution for this is the use a likelihood based on a log-normal distribution:

\begin{equation}
    \mathcal{L}(\Delta \mu) = \prod_{i=1}^{N_{\textrm{DOM}}} \frac{1}{\sigma_{i} r_{i} \sqrt{2\pi} } \textrm{exp} \left\{ -\frac{1}{2} \left( \frac{ln(r_{i}) - (\mu_{i} + \epsilon_{i}\Delta \mu)}{\sigma_{i}}  \right)^{2} \right\}
\end{equation}

\noindent where 

\begin{equation}
  \mu_{i} = ln\left( \frac{m^{2}}{\sqrt{ v + m^{2}}}\right)
\end{equation}

\noindent and

\begin{equation}
  \sigma_{i} = \sqrt{ ln(1 + \frac{v}{m^{2}}) }
\end{equation}

\noindent $m$ and $v$ are the mean and variance of the data, respectively. This changes $\Delta \mu$, such that

\begin{equation}
  \Delta \mu = \sigma_{\Delta \mu} \sum_{i = 1}^{N_{\textrm{DOM}}} \frac{\epsilon_{i}}{\sigma_{i}} (ln(r_{i}) - \mu_{i})
\end{equation}
