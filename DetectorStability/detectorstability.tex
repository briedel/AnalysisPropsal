\section{Detector Stability}

Analysis of the detector stability using of partial detectors, 40-string though 79-string configuration, in \cite{vbaumaster} and \cite{mkrasbergtalk} showed that there was significant rise in noise that could be attributed to the recently deployed strings and the on going freeze-in process. For the full detector, the same effect is seen, see Figure~\ref{fig:snscalerIC86ItoIII}. The freeze-in process had an significant effect on the first half of the first full year of data taking, see Figure~\ref{fig:snscalerIC86I}. As the freeze-in process subsides, the annular modulation of the muon rate with temperature, see Figure~\ref{fig:muonseasonal}, causes the detector rate to have a sinusoidal-like time dependence. This becomes the most significant contribution to detector instability over time. 

Figure~\ref{fig:snscalerIC86ItoIII} also reveals a continual decay in the detector rate throughout the operation of the full detector configuration. Comparison of supernova scaler rate for the 86-string configuration for strings deployed at all the different construction cycles, see Figure~\ref{fig:scalerrateage}, shows that the decay is on-going even several years after deployment. Similarly, the freeze-in progresses has a depth dependence the, see Figure~\ref{fig:scalerratedepth}. The origin of this decay is unknown. There are multiple possible explanations that can be examined from the data:

\begin{itemize}
	\item Construction effects: Continuation of the freeze-in process and larger amount of encased particles from construction
	\item Detector effects: Detector aging and drift of detector settings away from optimal values
\end{itemize}

The freeze-in process continues even after the ice appears to be solid, as shown by the ``Swedish Camera'', a camera system deployed in the deep ice at the end of String 80. Recordings from the ``Swedish Camera'', see Figure~\ref{fig:swedencamera}, have shown that the ice surrounding the bore hole continues to change over the years. The air and particles trapped in the ice from drilling and refilling the bore holes are being pressed into a central column. This processes appears to be on-going and may cause low levels of triboluminescence similar to the triboluminescence caused by the solidification process that are seen as a decay in the supernova scaler. In order to verify this, a comparison between the freeze-in rate established from ``Swedish Camera'' data and the decay rate seen at the bottom of the detector should be performed.

This comparison the decay rate is associated with the freeze-in process surrounding the bore hole and the possible triboluminescence caused by this. To establish the decay rate for the detector rate, a fit of a decaying exponential:

\begin{equation}
  R(t) = A + B \times \textrm{exp}\left( -\tau t \right)   
\end{equation} 

\noindent is performed to the area of the data where the freeze-in dominates and where the muon background is minimal, see Figure~\ref{fig:snscalerIC86ItoIIIannotated}. This fit assumes that the sinusoidal-like effect produced by the atmospheric muons averages out over time and is absorbed into the term for the average Poisson rate, $A$. A fit to the data in the dust layer would only 