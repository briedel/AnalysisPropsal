\section{Detector Stability}

Analysis of the detector stability using of partial detectors, 40-string though 79-string configuration, in \cite{vbaumaster} and \cite{mkrasbergtalk} showed that there was significant rise in noise that could be attributed to the recently deployed strings and the on going freeze-in process. For the full detector, the same effect is seen, see Figure~\ref{fig:snscalerIC86ItoIII}. The freeze-in process had an significant effect on the first half of the first full year of data taking, see Figure~\ref{fig:snscalerIC86I}. As the freeze-in process subsides, the annular modulation of the muon rate with temperature, see Figure~\ref{fig:muonseasonal}, causes the detector rate to have a sinusoidal-like time dependence. This becomes the most significant contribution to detector instability over time. 

\begin{figure}
  
\end{figure}

Figure~\ref{fig:snscalerIC86ItoIII} also reveals a continual decay in the detector rate throughout the operation of the full detector configuration. Comparison of supernova scaler rate for the 86-string configuration for strings deployed at all the different construction cycles, see Figure~\ref{fig:scalerrateage}, shows that the decay is on-going even several years after deployment. Similarly, the freeze-in progresses has a depth dependence the, see Figure~\ref{fig:scalerratedepth}. The origin of this decay is unknown. There are multiple possible explanations that can be examined from the data:

\begin{itemize}
  \item Construction effects: Continuation of the freeze-in process
  \item Detector effects: Detector aging and drift of detector settings away from optimal values
\end{itemize}

The freeze-in process continues even after the ice appears to be solid, as shown by the ``Swedish Camera'', a camera system deployed in the deep ice at the end of String 80. Recordings from the ``Swedish Camera'', see Figure~\ref{fig:swedencamera}, have shown that the ice surrounding the bore hole continues to change over the years. The air and particles trapped in the ice from drilling and refilling the bore holes are being pressed into a central column. This processes appears to be on-going and may cause low levels of triboluminescence similar to those associated with the solidification process that are seen as a decay in the supernova scaler, see Figure~\ref{fig:snscalerIC86I}. 

To establish the decay rate for the individual DOMs, a fit of a decaying exponential:

\begin{equation}
  R(t) = A + B \times \textrm{exp}\left( -\tau t \right)   
\end{equation} 

\noindent is needed to the scaler data. The sinusoidal-like effect of the atmospheric muons has to be removed before such a fit can be produced. The IceCube trigger-level data has a significantly higher energy threshold and lower rate than the scaler data, i.e. $\approx \unit[2100]{Hz}$ versus $\approx \unit[5200]{Hz}$. To estimate the atmospheric muon effect with a lower energy threshold, the HLC rate for individual DOMs from the pDAQ log-files is extracted. The pDAQ log-files show a HLC rate of $\mathcal{O}(\unit[10]{Hz})$. This HLC rate mimics the effect of the atmospheric muons, as only two effects can set off the HLC condition: noise triggers on adjacent DOMs or by-passing muons. The coincidence rate, $R_{C}$, between two DOMs, assuming a noise rate of \unit[500]{Hz} for each DOM, is given by

\begin{eqnarray}
  R_{C} &=& R_{1} R_{2} 2\Delta t \\ &=& (\unit[500]{Hz})(\unit[500]{Hz})(\unit[2 \times 10^{-6}]{s}) \\ &=& \unit[0.5]{Hz}
\end{eqnarray}

\noindent where $R_{1}$ is the rate of DOM 1, $R_{2}$ is the rate of DOM 2, and $\Delta t$ is the possible time window which these two can overlap, in this case the HLC condition window of \unit[1]{$\mu$s}. In case of the HLC condition we have to take the rates 5 DOMs into consideration, i.e. 2 DOMs above and 2 DOMs below. This means the coincidence rate is given by

\begin{equation}
  R_{C} $=$ 2 \Delta t R_{3}\left( R_{1} + R_{2} + R{4} + R{5} \right)
\end{equation}

\noindent where $R_{3}$ is the rate of the central DOM, and $R_{1, 2, 4, 5}$ is the rate of the surrounding DOMs. The noise rate for each DOM can be estimated using the SLC rate and the SN scaler rate for each DOM. The coincidence rate to satisfy the HLC condition can be approximated as to \unit[2]{Hz}. The remaining rate can be attributed to the effect of atmospheric muon. 

The SN scalers have a lower energy threshold than HLC though as they are sensitive to individual hits. The rate of individual hits attributed to atmospheric muons has to estimated from simulation. The single hit rate in data can be estimated by simulating muons with a known spectrum and determining the per-DOM single hit and HLC rate. The per-DOM single hit rate can than be scaled according to the ratio of the per-DOM HLC rate between simulation and data. 

% This fit assumes that the sinusoidal-like effect produced by the atmospheric muons averages out over time and is absorbed into the term for the average Poisson rate of DOM noise and atmospheric muons, $A$. A fit to the data from the dust layer would suppress the effect of the muon background, see Figure~\ref{fig:scalerratedepth} Panel for DOMs at position 35, but would not capture the depth dependence of the freeze-in process. 

 % In order to verify this, a comparison between the freeze-in rate established from ``Swedish Camera'' data and the decay rate seen at the bottom of the detector should be performed.

% This comparison would show whether the decay rate is associated with the freeze-in process surrounding the bore hole and its possible triboluminescence, as these two processes are causally connected. To establish the decay rate for the detector rate, a fit of a decaying exponential:

% \begin{equation}
%   R(t) = A + B \times \textrm{exp}\left( -\tau t \right)   
% \end{equation} 

% \noindent will be performed to different layers of the ice. This fit assumes that the sinusoidal-like effect produced by the atmospheric muons averages out over time and is absorbed into the term for the average Poisson rate of DOM noise and atmospheric muons, $A$. A fit to the data from the dust layer would suppress the effect of the muon background, see Figure~\ref{fig:scalerratedepth} Panel for DOMs at position 35, but would not capture the depth dependence of the freeze-in process. 

% In case the assumption for the approach outlined above breaks down or an adequate goodness-of-fit cannot be established, there are two more possible approaches: using the dust layer suppression of the atmospheric muon modulation to correlate the decay rate to ice temperature and removing the atmospheric muon background by estimating the average muon rate the supernova scalers are sensitive to from a comparison of simulation and data. 

% The freeze-in rate is associated with depth, see Figure~\ref{fig:scalerratedepthfreezein}. The depth is correlated to the temperature of the ice surrounding the bore hole. Given the suppression of the muon rate in the dust layer mentioned above, one could establish a correlation between temperature in the dust layer and the detector decay rate. This correlation can be extrapolated to the temperature range seen in the detector.

% Estimating the atmospheric muon rate that effect the supernova scalers is impossible solely using triggered IceCube data or simulation. The IceCube trigger-level data has a significantly higher energy threshold and lower rate than the scaler data, i.e. $\approx \unit[2100]{Hz}$ versus $\approx \unit[5200]{Hz}$. There are also known discrepancy between data and simulation, this is mostly due not-well-constrained hardonic interaction models for cosmic ray interactions and detector systematics, such as the ice. In order to be able to estimate the average muon rate that the supernova scalers are sensitive to, a comparison between the average muon rate at IceCube trigger-level between simulation and data at various nchannels has to be performed. The ratio between the two average muon rate for various nchannel between data and simulation will yield a scaling factor that can be applied to the sub-trigger-level average muon rate as a function of nchannel. This estimate of the average muon rate can than be used to determine the correlation between effective atmospheric temperature and muon rate, which makes it possible to subtract an estimate of the average muon rate from the data given the atmospheric temperature.

Effects from detector aging would be hard to disentangle from difference in the DOM components between different revisions of the DOM hardware and possible freeze-in effects. Further long-term studies with the DOMs in the freezer at Physical Science Laboratory would be necessary.

A drift of detector settings away from optimal values should be detectable from changes in the time constant of supernova scaler decay rate between strings deployed at the different construction cycles and during physics run transitions, respectively. A drift away from optimal values should be detectable by a jump or sink in the supernova scaler during the run transition. A jump in the detector rate was found during the physics run transitions, see Figure~\ref{fig:snscalerIC86ItoIII}. This jump can be attributed to the addition of previously removed DOMs from the detector configuration. There are no jump or sinks in the rate for the strings in Figure~\ref{fig:scalerrateage} during run transition leading to the conclusion that the effect of the new calibration values in minimal. 


