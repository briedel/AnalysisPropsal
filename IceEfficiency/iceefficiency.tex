\section{Ice Efficiency}

The optical properties of the ice in which IceCube is embedded are not uniform throughout the detector. One of the main assumptions that goes into the supernova onlien analysis a uniform illumination of the ice by the interaction of supernova neutrinos. The different optical properties however will cause the ice to not be illuminated uniformly, as some DOMs will have greater sensitivity than others because the light from the products of supernova neutrino interactions can travel more readily through the ice.

From studying the ice properties versus depth for MeV-scale positrons, one can establish an efficiency factor for the ice surrounding the DOM. To do this, an average of the effective volume of the DOMs outside the dust layer is calculated for an IceCube. The DOMs in the dust layer are excluded from the average because of the poor optical properties of the ice and the articifical decrease of the average effectice volume this would produce. The ratio between the average and the effective volume for the DOM is defined as the ice efficiency parameter. The effective volume of the DeepCore DOMs has to adjusted from the higher quantum efficiency as this effect is already taken into account separately, but still be able to compensate for their different location inside the detector. 